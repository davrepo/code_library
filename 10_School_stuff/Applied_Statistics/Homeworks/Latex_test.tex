\documentclass[20pt]{extbook}
\begin{document}


# Problems

T=Theoretical Exercise, R=R-Exercise

## 1. Sum of Bernoulli Distributed Variables (T)

Let $X$ and $Y$ be two independent random variables where $X \sim \mathrm{Ber}(p)$ and $Y \sim \mathrm{Ber}(q)$. Let $Z=X+Y$. Investigate how $Z$ is distributed by deriving the probability mass function for $Z$.

To derive the probability mass function for $Z$, we need to consider all possible values of $Z$ and calculate their probabilities.

Since $X$ and $Y$ are independent, we can use the fact that the probability of the intersection of two independent events is the product of their individual probabilities.

For $Z = 0$, we have:

$$P(Z = 0) = P(X + Y = 0) = P(X = 0 \text{ and } Y = 0) \\ = P(X = 0) \cdot P(Y = 0) 
= (1 - p) \cdot (1 - q)$$

\binom{}{}   \int_{}^{} \bullet \cdot  \,dx 

For $Z = 1$, we have:

$$P(Z = 1) = P(X + Y = 1) = P(X = 1 \text{ and } Y = 0) + P(X = 0 \text{ and } Y = 1) \\ = P(X = 1) \cdot P(Y = 0) + P(X = 0) \cdot P(Y = 1) = p \cdot (1 - q) + (1 - p) \cdot q$$

For $Z = 2$, we have:

$$P(Z = 2) = P(X + Y = 2) = P(X = 1 \text{ and } Y = 1) \\ = P(X = 1) \cdot P(Y = 1) = p \cdot q$$

Therefore, the probability mass function for $Z$ is:

$$P(Z = 0) = (1 - p) \cdot (1 - q)$$ $$P(Z = 1) = p \cdot (1 - q) + (1 - p) \cdot q$$ $$P(Z = 2) = p \cdot q$$

Note that this distribution is called the binomial distribution with parameters $n = 2$ and $p$, where $p$ is the probability of success in a single Bernoulli trial.

## 2. Law of large numbers (T)

Assume that you want to simplify your bookkeeping by rounding your expenses to nearest €. Let the rounding error of a single purchase be described by the random variable $X$. Assume that $n$ is the number of purchases you have made during a year.

(a) How is $X$ distributed?

The rounding error $X$ is the difference between the actual expense and the rounded expense. Assuming that the rounding is to the nearest Euro, $X$ can take on values between $-0.50$ Euro and $+0.50$ Euro.

If we assume that the rounding errors of different purchases are independent and identically distributed, and the probability of rounding up or down is equal (i.e., a fair rounding), then $X$ can be modeled as a discrete uniform distribution with the following probabilities:

\begin{align*}
P(X=-0.50) &= P(X=0.50) = \frac{0.5}{n} \
P(X=0) &= \frac{n-1}{n}
\end{align*}

This is because there are only two possible outcomes for $X$, and each outcome has an equal probability of occurring (except for the middle value, 0, which has a slightly higher probability due to the way rounding works).

Note that the assumption of independent and identically distributed rounding errors may not always hold in practice, and the actual distribution of $X$ may vary depending on the specific circumstances.




(b) What is the expected rounding error for a single purchase?
(c) What happens to the mean absolute error $\frac{1}{n} \sum_{i=1}^n |X_i|$ when $n$ grows in the light of the law of large numbers?

## 3. Central Limit Theorem (T)

Let $X_1, X_2,\ldots$ be a sequence of independent, identically distributed random variables, with probability density function given by \begin{equation}
f(x) = \left\{\begin{tabular}{ll} $x$, & if $0<x\leq1$ \\ 
                                  $-x + 2$, & if $1<x<2$ \\
                                  0, & otherwise. \end{tabular}\right.
\end{equation} Use central limit theorem to approximate $P(X_1 +X_2 + \ldots + X_{277} > 301)$.

\cdot \sum_{n = 1}^{\infty} \rightarrow  

\end{document}\cap \binom{}{} 
\sigma \neq 
\leq \approx \thicksim \pm 